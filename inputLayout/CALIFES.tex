%% TIKZ DRAWING OF CALIFES ELEMENTS
%% TO BE INCLUDED INTO A LATEX DOCUMENT
%% INSIDE A TIKZPICTURE ENVIRONMENT
%% K. Sjobak, 2018--2020, D. Gamba 2020, L.A. Dyks 2021, 
%% A. Gilardi 2025
%% G. Tangari - S. Simonsson 2025

    % Initialise some variables: %%%%%%%%%%%%%%%%%%%%%%%%%%%%%%%
    \pgfmathsetmacro{\VesperStart}{1.5}; %Center of the dipole (X)
    \pgfmathsetmacro{\NewBemStart}{\VesperStart}    % split at same X as VESPER
    \pgfmathsetmacro{\NewBemAngle}{22}   % mirror angle (positive)
    \pgfmathsetmacro{\NewBemX}{-1}
    \pgfmathsetmacro{\NewBemY}{(\NewBemStart-\NewBemX)*tan(\NewBemAngle)}

    \pgfmathsetmacro{\VesperAngle}{-15.0};
    %Arrow end point
    \pgfmathsetmacro{\VesperX}{-1};
    \pgfmathsetmacro{\VesperY}{(\VesperStart-\VesperX)*tan(\VesperAngle)};

    \pgfmathsetmacro{\VesperTabX}{-0.0};
    \pgfmathsetmacro{\VesperTabY}{(\VesperStart-\VesperTabX)*tan(\VesperAngle)};

    \if\getvalue{distances}1
        \pgfmathsetmacro{\RFyOFF}{-0.0};
    \else
        \pgfmathsetmacro{\RFyOFF}{-0.5};
    \fi

    \if\getvalue{elemnames}1
        \pgfmathsetmacro{\BotLegyOFF}{0.0};
    \else
        \pgfmathsetmacro{\BotLegyOFF}{1.1};
    \fi
    %%%%%%%%%%%%%%%%%%%%%%%%%%%%%%%%%%%%%%%%%%%%%
    %% GREEN Area to indicate experiments
    %%%%%%%%%%%%%%%%%%%%%%%%%%%%%%%%%%%%%%%%%%%%%

    %EOS table
    \filldraw[green] (3.55, -0.5) rectangle (3.2,0.5);
    \node[above] at (3.45,0.95) {\tiny CAMEOS1};
    \node[above] at (3.45,0.65) {\tiny CAMEOS2};
    \node at (3.45, -2.7+\BotLegyOFF) {\textbf{EOS}};

    \filldraw[green,rotate around={-\VesperAngle:(\VesperTabX,\VesperTabY)}]
    (\VesperTabX-0.5, \VesperTabY-0.4) rectangle
    (\VesperTabX+0.5, \VesperTabY+0.4);
    \node[above] at (\VesperTabX+0.6,0.95) {\tiny CAMVESPER1};
    %\node[rotate=-90,anchor=west] at (\VesperTabX,\VesperTabY-0.7) {\belemsiz VESPER};
    \node at (\VesperTabX+0.6, -2.7+\BotLegyOFF) {\textbf{VESPER}};

    %%%%%%%%%%%%%%%%%%%%%%%%%%%%%%%%%%%%%%%%%%%%%%%
    %%%%%%%%%%%%%%%%%%%%%%%%%%%%%%%%%%%%%%%%%%%%%%%
    %%%%%%%%%%%%%%%%%%%%%%%%%%%%%%%%%%%%%%%%%%%%%%%


    %% BEAM
    \if\getvalue{unbroken}0
        \draw[latex-,ultra thick] (0,0)--(15,0);
        \draw[ultra thick, dashed] (-1,0) -- (0,0);
    \else
        \draw[latex-,ultra thick] (-1,0)--(15,0);
    \fi

    %% GUN
    \if\getvalue{distances}1
        \node at (15,1) {\belemsiz 0.0~m};
    \fi

    % Gun solenoids
    \solRect{15}{0.25}{14.25}{0.35}
    \solRect{15}{-0.25}{14.25}{-0.35}

    \solRect{15.25}{0.25}{15}{0.35}
    \solRect{15.25}{-0.25}{15}{-0.35}

    % Gun cavity
    \draw[orange,  thick] (15,0.0) to (15,0.2)
        arc(90:180:0.1)
        arc(360:180:0.05) arc(0:180:0.1)
        arc(360:180:0.05) arc(0:180:0.1);
    \draw[orange, thick] (15,0.0) to (15,-0.2)
        arc(-90:-180:0.1)
        arc(0:180:0.05) arc(0:-180:0.1)
        arc(0:180:0.05) arc(0:-180:0.1);

    \if\getvalue{elemnames}1
        \node[rotate=-90,anchor=west,align=left] at (15+0.25/2, -0.7)
            {\belemsiz SNH 110};
        \node[rotate=-90,anchor=west,align=left] at (14.5, -0.7)
            {\belemsiz GUN 115\\
             \belemsiz SNI 120};
    \fi

    \correctorMagnet{13.7}{D\%G 130};
    \if\getvalue{distances}1
        \node at (13.7,1) {\belemsiz 0.32~m};
    \fi

    \if\getvalue{auxElems}1
        % Laser mirror
        \draw[ultra thick]  ({(13.7-13)/2+13 + 0.1}, -0.1) --
                        ({(13.7-13)/2+13 - 0.1}, -0.3);

        %Laser
        \draw[thick,-latex,dashed]
            ({(13.7-13)/2+13},-2.25+\BotLegyOFF) --
            ({(13.7-13)/2+13},-0.2) --
            (15,0);

        %Laser table
        \BTV[-2.25+\BotLegyOFF]{14}{};
        \draw[thick, -latex, dashed]
            (13,-2.25+\BotLegyOFF) -- (14,-2.25+\BotLegyOFF);
        \if\getvalue{elemnames}1
            \node at (14.9,-2.25) {\belemsiz BTV125};
        \fi

        \draw[ultra thick] ({(13.7-13)/2+13-0.25/2}, -2.25-0.25/2+\BotLegyOFF) --
                           ({(13.7-13)/2+13+0.25/2}, -2.25+0.25/2+\BotLegyOFF);
    \fi

    %ICT 210
    \filldraw[purple] (13.05-0.1/2,-0.3) rectangle (13.05+0.1/2,0.3);
    \if\getvalue{elemnames}1
        \node[rotate=-90,anchor=west] at (13.05,-0.7) {\belemsiz ICT 210};
    \fi

    \BTV{12.65}{MTV 215};
    \if\getvalue{distances}1
        \node at (12.65,0.6) {\belemsiz 1.81~m};
    \fi

    \cBPM{12.15}{BPC 220}{0.15};
    \correctorMagnet{11.9}{D\%G 225}; 
    %\node at (11.5,1) {\belemsiz 2.18~m}; !Moved to the bottom, to be drawn on top of RF network

    %ACS 230
    \filldraw[ultra thick, orange] (11.65,-0.15) rectangle (10.55,0.15);

    \solRect{11.65}{0.3}{10.55}{0.4}
    \solRect{11.65}{-0.3}{10.55}{-0.4}

    \filldraw[blue] (11.65, 0.2) rectangle (10.55,  0.25);
    \filldraw[blue] (11.65,-0.2) rectangle (10.55, -0.25);

    \if\getvalue{elemnames}1
        \node[rotate=-90,anchor=west, align=left] at (10.55+1.1/2,-0.7)
            {\belemsiz ACS 230\\
             \belemsiz DB 230-S\\
             \belemsiz SNG 230};
    \fi

    \BTV{10.15}{MTV 235};
    \if\getvalue{distances}1
        \node at (10.15,0.6) {\belemsiz 0.15~m};
    \fi
    
    \cBPM{9.65}{BPC 240}{0.15};
    \correctorMagnet{9.4}{D\%G 245};
    %\node at (9.6,1) {\belemsiz 7.33~m}; !Moved to bottom

    % ACS 250
    \filldraw[ultra thick, orange] (9.2,-0.15) rectangle (8.1,0.15);
    \if\getvalue{elemnames}1
        \node[rotate=-90,anchor=west, align=left] at (8.1+1.1/2,-0.7) {\belemsiz ACS 250\\
                                                                       \belemsiz SNG 250};
    \fi

    \solRect{9.2}{0.2}{8.1}{0.3};
    \solRect{9.2}{-0.2}{8.1}{-0.3};

    \cBPM{7.9}{BPC 260}{0.15};
    \correctorMagnet{7.6}{D\%G 265};
    %\node at (7.6,1) {\belemsiz 12.49~m}; !Moved to bottom

    % ACS 270
    \filldraw[ultra thick, orange] (7.3,-0.15) rectangle (6.2,0.15);
    \if\getvalue{elemnames}1
        \node[rotate=-90,anchor=west] at (6.2+1.1/2,-0.7) {\belemsiz ACS 270};
    \fi

    %\node at (10,-2.25) {\textbf{CALIFES S-band injector}};

    \cBPM{5.9}{BPC 310}{0.15};
    \correctorMagnet{5.6}{D\%G 320};
    \if\getvalue{distances}1
        \node at (5.6,1) {\belemsiz 17.70~m};
    \fi

    \kickerHV[0]{5.125}{SDH 340}{-1}{orange};

    %\draw[ultra thick, purple] (4.15,-0.15) rectangle (3.85,0.15);

    \lensF{4.6}{QFD 350};
    \lensD{4.2}{QDD 355};
    \lensF{3.8}{QFD 360};
    %\node at (3.9,1) {\belemsiz 18.90~m};
    % unknown position after EOS installation
    \if\getvalue{distances}1
        \node at (4.1,1) {\belemsiz $\approx$19~m};
    \fi

    %%EOS table defined on top, to get behind the "beam"
    %\draw[-latex, dashed,thick] (4.8,-2) -- (4.8,-0.2) -- (3.0,-0.2) -- (3,-2);
    %%\node at (4.8,-2.25) {\belemsiz EOS laser};
    %\node at (3.9,-2.25) {\textbf{Electro-Optical Sampling}};
    \filldraw[purple!70] (3.3,-0.4) rectangle (3.45,0.4);
    \if\getvalue{elemnames}1
        \node[rotate=-90,anchor=west,align=left] at (3.375,-0.7)
            {\belemsiz EOS};
    \fi

    \cBPM{2.975}{BPC 380}{0.15};
    \correctorMagnet{2.675}{D\%G 385};
    %\node at (2.4,1) {\belemsiz 19.96~m};
    % unknown position after EOS installation
    \if\getvalue{distances}1
        \node at (2.55,1) {\belemsiz $\approx$20~m};
    \fi

    \BTV{2.25}{MTV 390};

    %ICT 395
    \filldraw[purple] (1.85-0.1/2,-0.3) rectangle (1.85+0.1/2,0.3);
    \if\getvalue{elemnames}1
        \node[rotate=-90,anchor=west] at (1.85,-0.7) {\belemsiz ICT 395};
    \fi

    %% VESPER


    \filldraw[gray,rotate around={-\VesperAngle:(\VesperX,\VesperY)}]
        (\VesperX-0.05, \VesperY-0.5) rectangle
        (\VesperX+0.05, \VesperY+0.5);

    \draw[->,ultra thick] (\VesperStart,0)--(\VesperX,\VesperY);

    % Arrowed solid part (first segment)
    \draw[->,ultra thick] (\NewBemStart,0) --
   ($(\NewBemStart,0)!0.8!(\NewBemX,\NewBemY)$);

    % Dashed continuation (second segment, no arrow)
    \draw[ultra thick, dashed] 
   ($(\NewBemStart,0)!0.6!(\NewBemX,\NewBemY)$) --
   (\NewBemX,\NewBemY);

    %\kickerHV[0]{\VesperStart}{BHB 100}{1}{blue};
    \dipole[0]{\VesperStart}{BHB 100}{{-\NewBemAngle/2}};

    \pgfmathsetmacro{\VesperBTVaX}{0.9};
    \pgfmathsetmacro{\VesperBTVaY}{(\VesperStart-\VesperBTVaX)*tan(\VesperAngle)};
    \BTV[\VesperBTVaY]{\VesperBTVaX}{BTV420}

    % ICT 430
    \pgfmathsetmacro{\VesperICTX}{0.4};
    \pgfmathsetmacro{\VesperICTY}{(\VesperStart-\VesperICTX)*tan(\VesperAngle)};
    \filldraw[purple,rotate around={-\VesperAngle:(\VesperICTX,\VesperICTY)}]
        (\VesperICTX-0.1/2, \VesperICTY-0.3) rectangle
        (\VesperICTX+0.1/2, \VesperICTY+0.3);

    \if\getvalue{elemnames}1
        \node[rotate=-90,anchor=west] at (\VesperICTX,\VesperICTY-0.7) {\belemsiz ICT 430};
    \fi

    \pgfmathsetmacro{\VesperBTVbX}{-0.5};
    \pgfmathsetmacro{\VesperBTVbY}{(\VesperStart-\VesperBTVbX)*tan(\VesperAngle)};
    \BTV[\VesperBTVbY]{\VesperBTVbX}{BTV440}

    %% RF SYSTEM %%
    \if\getvalue{auxElems}1
        % MKS15
        \filldraw[orange!65] (12.75,1.0+\RFyOFF) -- (12.75+0.5,2+\RFyOFF) -- (12.75-0.5,2+\RFyOFF) -- cycle;
        \draw[-latex,orange!65,ultra thick] (12.75,1.75+\RFyOFF) -- (11.65,1.75+\RFyOFF) -- (11.65,0.4);

        \filldraw[orange!50] (11.95,1.75+\RFyOFF) circle (0.25);
        \if\getvalue{elemnames}1
            \node at (11.95,1.75+\RFyOFF) {\belemsiz $\Delta \Phi$};
        \fi

        \draw[-latex,orange!65,ultra thick] (12.65,1.75+\RFyOFF) -- (14.8+0.15/2,1.75+\RFyOFF) -- (14.8+0.15/2,0.4);

        \filldraw[orange!50] (14.3,1.75+\RFyOFF) circle (0.25);
        \if\getvalue{elemnames}1
            \node at (14.3,1.75+\RFyOFF) {\belemsiz $\Delta \Phi$};
        \fi

        \filldraw[orange!50] (13.55,1.75+\RFyOFF) circle (0.25);
        \if\getvalue{elemnames}1
            \node at (13.55,1.75+\RFyOFF) {\belemsiz $\Delta P$};
        \fi

        \if\getvalue{elemnames}1
            \node[align=center,anchor=north] at (12.75,2+\RFyOFF) {\belemsiz MKS\\ \belemsiz 15};
        \fi

        % MKS11
        \filldraw[orange!65] (10.7,1.0+\RFyOFF) -- (10.7+0.5,2+\RFyOFF) -- (10.7-0.5,2+\RFyOFF) -- cycle;
        \draw[orange!65, ultra thick, -latex] (10.5,1.75+\RFyOFF) -- (9.2,1.75+\RFyOFF) -- (9.2,0.3);
        \draw[orange!65, ultra thick, -latex] (10.5,1.75+\RFyOFF) -- (7.3,1.75+\RFyOFF) -- (7.3,0.2);

        \if\getvalue{elemnames}1
            \node[align=center,anchor=north] at (10.7,2+\RFyOFF) {\belemsiz MKS\\ \belemsiz 11};
        \fi

        % MKS31
        \filldraw[orange!65] (6.4,1.0+\RFyOFF) -- (6.4+0.5,2+\RFyOFF) -- (6.4-0.5,2+\RFyOFF) -- cycle;
        \draw[orange!65, ultra thick, -latex] (6.4,1.75+\RFyOFF) -- (5.125,1.75+\RFyOFF) -- (5.125,0.35);
        % \draw[orange, ultra thick, -latex] (10.5,1.75) -- (7.3,1.75) -- (7.3,0.4);

        \if\getvalue{elemnames}1
            \node[align=center,anchor=north] at (6.4,2+\RFyOFF) {\belemsiz MKS\\ \belemsiz 31};
        \fi
    \fi

    % Text on top of the orange
    \if\getvalue{distances}1
        \node at (11.5,1) {\belemsiz 2.18~m};
        \node at (9.6,1) {\belemsiz 7.33~m};
        \node at (7.6,1) {\belemsiz 12.49~m};
    \fi

%%% Local Variables:
%%% mode: latex
%%% TeX-master: "../layout.tex"
%%% End:

%% =========================
%% NEWBEMLINE 
%% =========================
% --- BPM on NewBemLine (same style as other BPMs) ---
%\pgfmathsetmacro{\NewBemBPMaX}{0.2};
%\pgfmathsetmacro{\NewBemBPMaY}{(\NewBemStart-\NewBemBPMaX)*tan(\NewBemAngle)};

% Draw BPM symbol, shifted to the branch Y. Pass empty label to suppress default text.
%\begin{scope}[shift={(0,\NewBemBPMaY)}]
%  \iBPM{\NewBemBPMaX}{}{0.15};
%\end{scope}

% Custom label above the element (only if names are enabled)
% \if\getvalue{elemnames}1
%  \node[anchor=south] at (\NewBemBPMaX,\NewBemBPMaY+0.30) {\belemsiz BPM 120};
% \fi
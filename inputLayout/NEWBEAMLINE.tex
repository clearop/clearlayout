%% TIKZ DRAWING OF CLEAR EXPERIMENTAL BEAMLINE 1 ELEMENTS
%% TO BE INCLUDED INTO A LATEX DOCUMENT
%% INSIDE A TIKZPICTURE ENVIRONMENT
%% A. Gilardi 2025
%% G. Tangari - S. Simonsson 2025


%% Source position 

\pgfmathsetmacro{\SrcX}{16} 
\pgfmathsetmacro{\SrcY}{-3} 

% Define outgoing angle (use your NewBemLine angle) 
\pgfmathsetmacro{\SrcAngle}{22}  % degrees 
\pgfmathsetmacro{\BreakX}{8.6}    % where it straightens out
\pgfmathsetmacro{\BreakY}{\SrcY + (\SrcX-\BreakX)*tan(\SrcAngle)}  % <-- missing

    \if\getvalue{elemnames}1
        \pgfmathsetmacro{\BotLegyOFF}{0.0};
    \else
        \pgfmathsetmacro{\BotLegyOFF}{1.3};
    \fi

    %%%%%%%%%%%%%%%%%%%%%%%%%%%%%%%%%%%%%%%%%%%%%
    %% GREEN Areas to indicate experiments
    %%%%%%%%%%%%%%%%%%%%%%%%%%%%%%%%%%%%%%%%%%%%%

    %In Vacuum test area
    \filldraw[green] (4.75,-0.5) rectangle (3.85,0.5);
    \node[above] at (4.3,0.5) {\tiny BTVVAC1};
    \node[align=center] at (4.3,-3+\BotLegyOFF) {
        \textbf{In Vacuum} \\ \textbf{test area}
    };


    %IN-AIR table
    \filldraw[green] (-0.6,-0.5) rectangle (0.65,0.5);
    \node[above] at (0,0.5) {\tiny CAMAIR1-4};
    \node[align=left,anchor=west] at (-0.7,-3+\BotLegyOFF) {\textbf{In-Air}\\ \textbf{test area II}};

    %%%%%%%%%%%%%%%%%%%%%%%%%%%%%%%%%%%%%%%%%%%%%%%
    %%%%%%%%%%%%%%%%%%%%%%%%%%%%%%%%%%%%%%%%%%%%%%%
    %%%%%%%%%%%%%%%%%%%%%%%%%%%%%%%%%%%%%%%%%%%%%%%

    % FINAL DUMP
    \if\getvalue{elemnames}1
        \filldraw[gray] (-0.6,-0.5) rectangle (-0.5,0.55);
    \else
        \filldraw[gray] (-0.6,-0.55) rectangle (-0.5,0.5);
    \fi



    % Dashed part first (no arrow)
    \draw[ultra thick, dashed] (\SrcX,\SrcY) --
   ($(\SrcX,\SrcY)!0.6!(\BreakX,\BreakY)$);

    % Solid continuation (still no arrow)
    \draw[ultra thick]
   ($(\SrcX,\SrcY)!0.0!(\BreakX,\BreakY)$) --
   (\BreakX,\BreakY);

    % Continue straight horizontal beamline from the break point, solid arrow
    \draw[->, ultra thick] (\BreakX,\BreakY) -- (-0.5,\BreakY);


    % direction of the diagonal (either works visually)
    \pgfmathsetmacro{\BeamDir}{-\SrcAngle}      % or: 180-\SrcAngle
    % --------------------------------------------


    

    \pgfmathsetmacro{\BPMx}{9.2}
    \pgfmathsetmacro{\BPMy}{\SrcY + (\SrcX-\BPMx)*tan(\SrcAngle)}
    \begin{scope}[shift={(0,\BPMy)}, rotate around={\BeamDir:(\BPMx,0)}] % BeamDir = -\SrcAngle
      \iBPM{\BPMx}{BPM 185};
    \end{scope}

    \pgfmathsetmacro{\CORx}{9.6}
    \pgfmathsetmacro{\CORy}{\SrcY + (\SrcX-\CORx)*tan(\SrcAngle)}
    \begin{scope}[shift={(0,\CORy)}, rotate around={\BeamDir:(\CORx,0)}]
    \correctorMagnet{\CORx}{D\%J 175};
    \end{scope}
    \if\getvalue{elemnames_newbeamline}1
        \node[rotate=-90] at (9.8, 1.0) {\belemsiz D\%J 175};
    \fi

    % --- Quadrupole F on the diagonal at x = 12.20 ---
    \pgfmathsetmacro{\QFx}{10.2}
    \pgfmathsetmacro{\QFy}{\SrcY + (\SrcX-\QFx)*tan(\SrcAngle)}
    \begin{scope}[shift={(0,\QFy)}, rotate around={\BeamDir:(\QFx,0)}]
      \lensF{\QFx}{QFG 170};
    \end{scope}
    \if\getvalue{elemnames_newbeamline}1
        \node[rotate=-90] at (10.4, 0.9) {\belemsiz QFG 170};
    \fi

    \pgfmathsetmacro{\SXx}{10.90}
    \pgfmathsetmacro{\SXy}{\SrcY + (\SrcX-\SXx)*tan(\SrcAngle)}
    \sextupole[\SXy]{\SXx}{XLA 165}{\BeamDir}
    \if\getvalue{elemnames_newbeamline}1
        \node[rotate=-90] at (11.10, 0.60) {\belemsiz XLA 165};
    \fi

    \pgfmathsetmacro{\QDx}{12.40}
    \pgfmathsetmacro{\QDy}{\SrcY + (\SrcX-\QDx)*tan(\SrcAngle)}
    \begin{scope}[shift={(0,\QDy)}, rotate around={\BeamDir:(\QDx,0)}]
      \lensD{\QDx}{QDG 155};
    \end{scope}
    \if\getvalue{elemnames_newbeamline}1
        \node[rotate=-90] at (12.6, -0.0) {\belemsiz QDG 155};
    \fi

    \pgfmathsetmacro{\CORx}{12.9}
    \pgfmathsetmacro{\CORy}{\SrcY + (\SrcX-\CORx)*tan(\SrcAngle)}
    \begin{scope}[shift={(0,\CORy)}, rotate around={\BeamDir:(\CORx,0)}]
    \correctorMagnet{\CORx}{D\%J 145};
    \end{scope}
    \if\getvalue{elemnames_newbeamline}1
       \node[rotate=-90] at (13.1, -0.20) {\belemsiz D\%J 145};
    \fi

    \pgfmathsetmacro{\SXx}{13.40}
    \pgfmathsetmacro{\SXy}{\SrcY + (\SrcX-\SXx)*tan(\SrcAngle)}
    \sextupole[\SXy]{\SXx}{XLA 135}{\BeamDir}
    \if\getvalue{elemnames_newbeamline}1
        \node[rotate=-90] at (13.60, -0.40) {\belemsiz XLA 135};
    \fi

    % --- Quadrupole F on the diagonal at x = 12.20 ---
    \pgfmathsetmacro{\QFx}{14.0}
    \pgfmathsetmacro{\QFy}{\SrcY + (\SrcX-\QFx)*tan(\SrcAngle)}
    \begin{scope}[shift={(0,\QFy)}, rotate around={\BeamDir:(\QFx,0)}]
      \lensF{\QFx}{QFG 130};
    \end{scope}
    \if\getvalue{elemnames_newbeamline}1
        \node[rotate=-90] at (14.20, -0.60) {\belemsiz QFG 130};
    \fi

    % --- MTV on the diagonal at x = 11.40 ---
    \pgfmathsetmacro{\MTVx}{15.00}
    \pgfmathsetmacro{\MTVy}{\SrcY + (\SrcX-\MTVx)*tan(\SrcAngle)}
     \begin{scope}[shift={(0,\BPMy)}, rotate around={\BeamDir:(\BPMx,0)}] % BeamDir = -\SrcAngle
        \BTV{\MTVx}{BTV 120};
    \end{scope}
    \if\getvalue{elemnames_newbeamline}1
        \node[rotate=-90] at (14.70, -0.90) {\belemsiz BTV 120};
    \fi
    
   % --- BPM on the diagonal at x = 12.00 ---
    % --- BPM 105 ---
    \pgfmathsetmacro{\BPMx}{15.40}
    \pgfmathsetmacro{\BPMy}{\SrcY + (\SrcX-\BPMx)*tan(\SrcAngle)}
    
    \begin{scope}[shift={(0,\BPMy)}, rotate around={\BeamDir:(\BPMx,0)}]
      \iBPM{\BPMx}{BPM 105};
    \end{scope}
    
    \if\getvalue{elemnames_newbeamline}1
      \node[rotate=-90] at (\BPMx,-1.3) {\belemsiz BPM 105};
    \fi
    
    
    % --- BHB 100, aligned on same tilted beamline ---
    \pgfmathsetmacro{\BHBx}{15.90}
    \pgfmathsetmacro{\BHBy}{\SrcY + (\SrcX-\BHBx)*tan(\SrcAngle)}
    
    \begin{scope}[shift={(0,\BHBy)}, rotate around={\BeamDir:(\BHBx,0)}]
      \dipole[0]{\BHBx}{BHB 100}{{0}};
    \end{scope}
    










    

    \dipole[0]{8.55}{BHB 200}{{0}};
    \if\getvalue{elemnames_newbeamline}1
        \node[rotate=-90] at (8.55,1.6) {\belemsiz BHB 200};
    \fi

    \BTV{7.95}{BTV 305};
    \if\getvalue{elemnames_newbeamline}1
        \node[rotate=-90] at (7.95,1.6) {\belemsiz BTV 305};
    \fi
    \correctorMagnet{7.5}{D\%J 310};
    \if\getvalue{elemnames_newbeamline}1
        \node[rotate=-90] at (7.5,1.6) {\belemsiz D\%J 310};
    \fi
    \if\getvalue{distances}1
        \node at (8.0,0.9) {\belemsiz 29.31~m};
    \fi
    
    \lensF{7.1}{QFG 315};
    \if\getvalue{elemnames_newbeamline}1
        \node[rotate=-90] at (7.1,1.6) {\belemsiz QFG 315};
    \fi
    \lensD{6.7}{QDG 320};
    \if\getvalue{elemnames_newbeamline}1
        \node[rotate=-90] at (6.7,1.6) {\belemsiz QFG 320};
    \fi
    \lensF{6.3}{QFG 325};
    \if\getvalue{elemnames_newbeamline}1
        \node[rotate=-90] at (6.3,1.6) {\belemsiz QFG 325};
    \fi
    \if\getvalue{distances}1
        \node at (5.95,0.9) {\belemsiz 30.62~m};
    \fi
    

    \iBPM{5.9}{BPM 330}{0.15};
    \if\getvalue{elemnames_newbeamline}1
        \node[rotate=-90] at (5.9,1.6) {\belemsiz BPM 330};
    \fi
    \correctorMagnet{5.6}{D\%J 335};
    \if\getvalue{distances}1
        \node at (5.2,1.15) {\belemsiz 31.66~m};
    \fi
    \if\getvalue{elemnames_newbeamline}1
        \node[rotate=-90] at (5.6,1.6) {\belemsiz D\%J 335};
    \fi

    \BTV{5.1}{BTV 420};
    \if\getvalue{elemnames_newbeamline}1
        \node[rotate=-90] at (5.1,1.6) {\belemsiz BTV 420};
    \fi

    \prescatt{3.45}{}; %1.65
    \if\getvalue{distances}1
        \node at (3.45,1.15) {\belemsiz 32.53~m};
    \fi
    % \if\getvalue{elemnames_newbeamline}1
    %     \node[rotate=-90] at (3.1,1.6) {\belemsiz PreSca?};
    % \fi

    \correctorMagnet{3}{D\%J 505};
    \if\getvalue{distances}1
        \node at (2.95,0.9) {\belemsiz 33.38~m};
    \fi
    \if\getvalue{elemnames_newbeamline}1
        \node[rotate=-90] at (3,1.8) {\belemsiz D\%J 505};
    \fi

    \iBPM{2.7}{BPM 510};
    \if\getvalue{elemnames_newbeamline}1
        \node[rotate=-90] at (2.7,1.85) {\belemsiz BPM 510};
    \fi

    \scatt{2.15}{BTV 520};
    \if\getvalue{distances}1
        \node at (2.5,1.85) {\belemsiz 32.53~m};
    \fi
        \if\getvalue{elemnames_newbeamline}1
        \node[rotate=-90] at (2.15,1.85) {\belemsiz BTV 520};
    \fi

    \draw[ultra thick, brown, {Latex[length=2.5mm]}-{Latex[length=2.5mm]}]
    (2.15,0.3) -- ++(0,0.8)
              -- ++(1.3,0)
              -- ++(0,-0.8);

    \lensF{1.7}{QFG 525};
        \if\getvalue{elemnames_newbeamline}1
        \node[rotate=-90] at (1.7,1.6) {\belemsiz QFG 525};
    \fi
    \lensD{1.3}{QDG 535};
        \if\getvalue{elemnames_newbeamline}1
        \node[rotate=-90] at (1.3,1.6) {\belemsiz QDG 535};
    \fi
    \lensF{0.9}{QFG 545};
        \if\getvalue{elemnames_newbeamline}1
        \node[rotate=-90] at (0.9,1.6) {\belemsiz QFG 545};
    \fi

        
    %% IN-AIR TABLE INSTRUMENTATION
    % \BTV{-0}{BTV 600};
    %     \if\getvalue{elemnames_newbeamline}1
    %     \node[rotate=-90] at (-0.1,1.6) {\belemsiz BTV 600};
    % \fi

    %ICT 
    \filldraw[purple] (0.4,-0.3) rectangle (0.5,0.3);
    \if\getvalue{elemnames}1
       \node[rotate = -90] at (0.45,-1.4) {\belemsiz (ICT 620)};
    \fi
            \if\getvalue{elemnames_newbeamline}1
        \node[rotate=-90] at (0.45,1.6) {\belemsiz ICT 620};
    \fi

